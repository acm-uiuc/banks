\documentclass{article}

\usepackage{newspaper}
\usepackage{times}
\usepackage{graphicx}
\usepackage{multicol}
\usepackage{picinpar}
%uasage of picinpar:
%\begin{window}[1,l,\includegraphics{},caption]xxxxx\end{window}


\date{October 31, 2025}
\currentvolume{43}
\currentissue{1}

\begin{document}
\maketitle

\begin{multicols}{3}

% Article 1 - Bold headline with byline
\byline{\textbf{\Large Local Cat Elected Mayor in Landslide Victory}}{Sarah Johnson}

In an unprecedented turn of events, Mr. Whiskers, a 5-year-old tabby cat, won the mayoral election yesterday with 73\% of the vote. His platform of ``more naps for everyone'' resonated deeply with voters.

The feline politician defeated incumbent Mayor Bob Thompson, who struggled to counter Mr. Whiskers' promise of mandatory afternoon quiet hours. ``I didn't see it coming,'' Thompson admitted. ``I thought my infrastructure plan would win, but apparently people prefer catnaps.''

Mr. Whiskers' campaign manager, 8-year-old Emma Rodriguez, said the victory proves that ``anyone can achieve their dreams, even if they sleep 16 hours a day.'' The mayor-elect was unavailable for comment, as he was napping in a sunny window.

\closearticle

% Article 2 - Italic headline
\headline{\textit{\huge Scientists Discover Coffee Actually IS a Food Group}}

Researchers at the Institute for Caffeinated Studies announced today that coffee should be reclassified as its own food group, separate from beverages. The groundbreaking study involved 10,000 participants who couldn't function before 10 AM.

\begin{window}[2,r,\includegraphics[width=0.9in]{atom.jpg},\centerline{\tiny Molecular Coffee}]
Dr. Janet Martinez, lead researcher, explained that the molecular structure of coffee exhibits properties of both liquid and spiritual sustenance. ``We found that coffee consumption increases productivity by 847\% and reduces homicidal thoughts by 92\%,'' she noted.

The FDA is now considering recommendations to update the food pyramid. Under the proposed changes, coffee would occupy the base layer, replacing grains and vegetables. ``This just makes sense,'' said FDA spokesperson Tom Chen.
\end{window}

Critics argue the study was funded by coffee companies, but researchers insist their findings are legitimate. ``We would never let corporate interests influence our work,'' Dr. Martinez said, sipping from her Grande Triple-Shot Espresso.

\closearticle

% Article 3 - Small caps headline
\headline{\sc\Large Time Traveler from 2045 Says We're Doing Everything Wrong}

A man claiming to be from the year 2045 appeared in the town square yesterday, shouting warnings about our current technology choices. John Smith, or ``Future John'' as locals call him, insists we need to stop several trends immediately.

``Delete all your social media accounts now!'' he screamed. ``In 2032, the Great Algorithm Wars begin, and it's not pretty.'' When asked for proof of his claims, Future John showed his phone, which he said was from 2045. It looked suspiciously like a current model iPhone.

Local authorities are evaluating his mental health. Future John was last seen trying to convince teenagers that vinyl records and flip phones are the technologies of tomorrow. ``You'll see,'' he muttered. ``You'll all see.''

\closearticle

% Article 4 - Sans serif bold headline
\headline{\textsf{\textbf{\Large Breaking: Nobody Knows How to Use Half the Features on Their Phone}}}

A comprehensive study released today reveals that 99.8\% of smartphone users have no idea what 50\% of their phone's features actually do. The study, conducted over three years, found that most people use their \$1,200 smartphones primarily for texting and occasionally taking photos.

``We interviewed thousands of users,'' explained researcher Dr. Lisa Park. ``Not a single person could explain what 'NFC' does, and only three people had ever used the calculator app for anything beyond splitting restaurant bills.''

The study also found that 87\% of apps installed on phones have never been opened after the first day. ``My phone says I have 247 apps,'' said study participant Mark Davidson. ``I use maybe 8 of them. The rest are just there, silently judging me.''

Tech companies responded by announcing plans to add 50 more features that nobody will use.

\closearticle

\end{multicols}

% Page 2
\newpage

\begin{multicols}{3}

% Article 5 - Large headline with subtitle effect
\byline{{\textit{\LARGE The Pizza Chronicles}}\\[8pt]{\Large Pineapple Debate Finally Settled}\\[5pt]}{Marcus Williams}

The centuries-old debate over pineapple on pizza has finally been resolved by the International Council of Pizza Affairs (ICPA). In a historic 47-page document released today, the council declared: ``Put whatever you want on your pizza. We're all adults here.''

The decision came after decades of heated arguments, family feuds, and at least three international incidents involving pineapple-topped pizzas. ``We're exhausted,'' said ICPA President Giuseppe Romano. ``Life is too short for this argument.''

However, the council did establish one firm rule: people must stop saying ``pizza is pizza'' to justify terrible toppings. ``If you put mayo and raisins on pizza, that's on you,'' Romano added. ``But don't call it good pizza.''

Social media erupted within minutes of the announcement, with both sides claiming victory. Meanwhile, pizzerias report that pineapple pizza sales remain exactly the same as before the ruling.

\closearticle

% Article 6 - Simple bold headline
\headline{\textbf{\Large Area Man Still Hasn't Watched Show Everyone Told Him About}}

Local resident David Park has successfully avoided watching the critically acclaimed series everyone has been recommending for three years. Despite constant pestering from friends, family, and coworkers, Park has maintained his position of ``I'll watch it eventually.''

``I'm sure it's great,'' Park said, scrolling through Netflix for the forty-fifth time this week. ``I just need to be in the right mood.'' He then proceeded to rewatch episodes of a sitcom he's seen seven times.

Park's friend Jessica Moore expressed frustration. ``We've all told him it's amazing. He's running out of excuses.'' Park's latest excuse, according to Moore, was that he ``doesn't want to commit to something with so many seasons.''

When asked what he's watching instead, Park admitted he just rewatches cooking competition shows while looking at his phone. ``It's different,'' he insisted.

\closearticle

% Article 7 - Mixed formatting headline
\headline{{\huge\textbf{Local Library Discovers}} \\[5pt] {\Large Forgotten Books Section That Nobody Knew Existed}}

Librarians at the Urbana Public Library made a shocking discovery this week: an entire section of the building that had been forgotten for approximately 30 years. The section, hidden behind a misplaced bookshelf, contains over 3,000 books.

``We were moving furniture for renovations,'' explained head librarian Patricia Evans, ``and we found a door. Behind it was a whole room full of books from the 1980s and 1990s.'' The collection includes complete sets of encyclopedias, outdated computer manuals, and several copies of books that were popular in 1992.

Most interesting, however, was the discovery of 47 overdue library books, some with late fees that have compounded to over \$15,000. The library has decided to waive the fees. ``What are we going to do, send collection agencies after someone for a book checked out in 1989?'' Evans laughed.

The forgotten section will be converted into a reading room, and some books will be added to the library's vintage collection. The computer manuals explaining Windows 3.1 will be recycled.

\closearticle

% Article 8 - Italic headline with different size
\headline{\textit{\LARGE Student Discovers Essay Writing Hack: Reading Assignment}}

College sophomore Amy Chen stumbled upon a revolutionary technique for writing better essays: actually reading the assigned material before writing about it. The discovery has shocked her professors and classmates alike.

``I used to just skim the first paragraph and make assumptions,'' Chen explained. ``But this semester I tried reading the whole thing. My grades went from C's to A's almost immediately.'' Chen's professor, Dr. Robert Collins, confirmed her improved performance. ``I was stunned. She actually quoted the text accurately. I almost cried.''

Other students are skeptical of Chen's method. ``That sounds like way too much work,'' said classmate Jake Morrison. ``I'll stick with my current system of reading SparkNotes ten minutes before class.''

Chen plans to share her discovery with others, though she's not optimistic about adoption rates. ``People say they'll try it, but then they see how many pages they have to read and immediately give up,'' she noted.

\closearticle

\end{multicols}

% Page 3
\newpage

\begin{multicols}{3}

% Article 9 - Professional headline
\byline{\textsc{\LARGE Technology Report}\\[8pt]{\Large New App Solves Problem Nobody Actually Had}}{Technology Desk}

Silicon Valley startup InnovateTech unveiled their latest app today, promising to revolutionize the way people track which hand they used to open doors. The app, called DoorHand, uses AI and machine learning to monitor and analyze door-opening patterns.

``We identified a gap in the market,'' said CEO Brandon Mitchell. ``Nobody was tracking their door-opening hand preferences. We're changing that.'' The app costs \$9.99 monthly and requires users to manually input data after opening each door.

Early reviews are mixed. Tech reviewer Sandra Lopez gave it 2 out of 5 stars, writing: ``The app works as described, but I'm not sure why I need it. After three days, I learned I'm right-handed, which I already knew.''

Despite lukewarm reception, InnovateTech has received \$50 million in venture capital funding. When asked about the business model, Mitchell explained they plan to ``disrupt the door-opening analytics space.'' Investors apparently found this convincing.

\closearticle

% Article 10 - Dramatic headline
\headline{\Huge\textbf{BREAKING:} \\[5pt] \Large Nobody Can Remember Their Password}

In a stunning development that surprised absolutely nobody, a global survey revealed that 94\% of people cannot remember their passwords and just click ``forgot password'' every time they need to log in.

The study, conducted by CyberSecurity Now, found that the average person has 87 online accounts and approximately 3 unique passwords that they rotate through. ``We found people using 'Password123' for everything from their bank to their Netflix account,'' said lead researcher Dr. Michael Torres.

When asked about password managers, most respondents said they have one but can't remember the master password. ``It's passwords all the way down,'' noted Dr. Torres. ``Some people have a password manager to remember the password for their other password manager.''

Security experts recommend using unique, complex passwords for each account. When told about this recommendation, survey respondents laughed and said they'll stick with their current system of writing passwords on Post-it notes stuck to their monitors.

\closearticle

% Article 11 - Sports section style
\headline{\textsf{\Large LOCAL TEAM DOES SPORT THING, FANS EXCITED}}

The Urbana Warriors did the sport thing successfully last night, defeating their rivals in a game that involved scoring points. Fans are reportedly very happy about the outcome.

``We really performed well,'' said team captain John Smith. ``We executed our strategy and managed to score more points than them.'' Smith's teammate, Mike Johnson, agreed: ``It was a good game. We did the thing.''

The victory marks the team's third consecutive win this season, putting them in a favorable position for the playoffs, which happen at some point in the future. Coach Robert Williams praised his team's effort. ``They played hard and followed the game plan. That's all you can ask for.''

Attendance at the game was described as ``pretty good'' with several people showing up. Concession sales were reportedly strong, with hot dogs being the most popular item.

\closearticle

% Article 12 - Opinion piece style
\byline{{\Large\textit{Opinion}}\\[8pt]{\large We Need to Talk About Daylight Saving Time}}{Editorial Board}

Twice a year, we all participate in a collective ritual of confusion, exhaustion, and complaining about clock changes. It's time to admit that Daylight Saving Time makes no sense anymore.

The original purpose was to save energy during World War I. That was over a century ago. We now have electric lights, and modern studies show the energy savings are negligible at best. What we do have is a nation of cranky, sleep-deprived people twice a year.

``But the extra daylight is nice,'' some say. Fine. Then pick one time and stick with it. Either stay on Daylight Saving Time year-round or stay on Standard Time. The switching is the problem, not the time itself.

Several states have already tried to opt out or make changes, but federal law makes this complicated. Congress needs to act. We've solved harder problems than this. We put people on the moon. Surely we can figure out how to stop changing our clocks.

Until then, we'll continue this bizarre tradition of pretending we can somehow create more daylight by moving our clocks around. See you in the spring when we all lose an hour of sleep again.

\closearticle

\end{multicols}

\end{document}
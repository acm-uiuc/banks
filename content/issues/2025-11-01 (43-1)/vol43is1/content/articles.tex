\needspace{5\baselineskip}

\label{article:banks}

\byline{\textbf{\Large Banks of the Boneyard Rises from the Dead!}}{Yanni Zhuang}

After a long hiatus, Banks of the Boneyard is back and better than ever! We're excited to finally bring back our beloved student-run publication. 

For those unfamiliar, Banks of the Boneyard (or just Banks for short) is a historically student-run publication managed by the ACM @ UIUC student organization. Its history can be traced back to the very inception of the ACM chapter itself, providing updates and expositions on the events and topics explored within the organization since 1984. Although Banks’ publication schedule has always been about as predictable as the Urbana weather, records indicate that publication ceased completely in 2009. A brief revival attempt was made in 2017 in the form of a Medium page, but that effort quickly fizzled out as well.

So why care? Why are we spending our precious time reviving what was already a shaky tradition in a medium long past its expiration date? Surely, there are better uses of our time. Well, there are a handful of reasons we felt that the revival of Banks was justified. 

Firstly, Banks was one of the only places to get a centralized view of what was happening within ACM. Past issues served as a conglomeration of updates from the overarching ACM organization, its SIGs, and its committees. For newcomers to ACM, it was a great way to get up to speed with the latest happenings. After the publication died out, this kind of information became progressively harder to find. Nowadays, past events only exist as notifications scattered across disjoint Discord servers and outdated websites. By bringing back Banks, we hope to make information about all the cool things our SIGs and committees are doing more accessible to all students. 

Secondly, in keeping with the idea that Banks was a centralized place for updates, it also serves as a snapshot of what ACM was doing at any given time. By looking at older Banks articles, we can piece together ACM’s history and evolution—from its humble beginnings as a small group of computing nerds seeking a community to tinker with, to the behemoth it is today: a very large group of computing nerds seeking a community to tinker with (but now with Google and LLMs). Being one of the oldest surviving technical organizations on campus, and being associated with an institution as deeply intertwined with the very founding of computing as a profession as Illinois, we ought to have a proper record of how students have engaged with the field. Unfortunately, as publication of Banks ceased in 2009, we have a knowledge gap in ACM history from 2009 to 2020. We hope that by bringing back Banks, we can maintain a written record of ACM's shenanigans for future generations to look back upon.


\closearticle


\needspace{5\baselineskip}

\label{article:rparticle}

\byline{\textbf{\Large Reflections \textbar{} Projections 2025 Recap}}{Cole Jordan}

\textbf{Reflections \textbar{} Projections 2025} is officially in the books! The Midwest’s largest student-run technology conference has concluded its 31st annual conference since it started in 1995. Hosted from \textbf{Tuesday, September 16th to Saturday, September 20th}, this year’s conference featured speakers from a wide range of fields in technology, including startups, autonomous vehicles, and artificial intelligence. We also hosted corporate events from HRT, Qualcomm, Aechelon, and Capital One, with the goal of providing students the opportunity to prepare for their future careers. As a conference made for students by students, we know what our attendees are looking for and design the entire conference around it.

\subsection*{The Name}

\textbf{Reflections \textbar{} Projections} has been the name of the conference since its inception in 1995; it was created to highlight the two main goals of the conference:
\begin{itemize}
\item \textbf{Reflect:} Take a look at the current state of technology and learn from innovators who have shaped the industry.
\item \textbf{Project:} Apply skills and knowledge towards the future by networking with companies and preparing for one’s career.
\end{itemize}


\subsection*{The Theme}

This year’s \textbf{R\textbar{}P theme} was inspired by \textbf{F1 and racecars}. In addition to our designs for our website, app, posters, and social media, we also decorated the first floor of Siebel to match our aesthetic. This theme proved incredibly successful, introducing many new faces to the conference and getting them excited about our events. We also introduced a \textbf{leaderboard} for the first time, where attendees earn points by attending events. The top 50 attendees on the leaderboard each day unlocked exclusive prizes, including a car keychain, a car stress toy, and a beanie. This was another way to encourage attendance at our events and provide attendees with unique prizes! Our speaker theme was also \textbf{“Racing Into The Future,”} with a specific emphasis on looking ahead to what the future of the industry will entail.

\subsection*{R\textbar{}P 2025 By The Numbers}

With 40+ staff members, 10+ months of preparation, 13 guest speakers, 12 sponsors, over 1500 attendees, over 800 meals handed out, and over 100,000 social media impressions, this year’s conference crossed the finish line at full speed!

\subsection*{Mobile App}

For the first time in R\textbar{}P history, our incredible development team deployed an official mobile app to streamline event operations. Users were able to flag events, receive push notifications, track their leaderboard standings, view food menus, and get scanned in for events, food, and swag all in one place.

\subsection*{Driving Innovation Showcase}

This year, we hosted R\textbar{}P’s very first \textbf{startup showcase}, allowing students to pitch their startup ideas to \textbf{Qualcomm engineers} to get feedback. We had over 10 groups pitch their innovative ideas, from deepfake detection to increasing AI accessibility.

\begin{center}\includegraphics[width=0.9\columnwidth]{./articles/images/driving_innovation_showcase.jpg}\end{center}
\textit{Students presenting their ideas during the Driving Innovation Showcase.}

\subsection*{Career Fair}

Our annual \textbf{career fair} brought hundreds of students to the first floor of the \textbf{Siebel Center for Computer Science}! Attendees had the chance to speak with representatives from companies like \textbf{HRT, Qualcomm, Caterpillar, Aechelon, Everfox, and Cloudflare}, and some attendees even landed internships and job offers from our career fair!

\begin{center}\includegraphics[width=0.9\columnwidth]{./articles/images/career_fair.jpg}\end{center}
\textit{Students connecting with recruiters at the R\textbar{}P 2025 Career Fair.}

\subsection*{MechMania and PuzzleBang}

R\textbar{}P includes two additional events every year: \textbf{MechMania}, a 24-hour AI hackathon, and \textbf{PuzzleBang}, a week-long puzzle hunt! Both of these events are organized to allow attendees to have some fun and utilize skills that they learned. \textbf{MechMania} was sponsored by Caterpillar this year and had record participation as teams built a bot to beat a soccer game. \textbf{PuzzleBang} coincided with our racing theme and created numerous themed puzzles and even a mechanic-themed escape box, all organized by UIUC alumni.


We are so proud of all the work our staff members put into making this year’s conference a success. While we might have reached the pitstop, we’re already turning the corner to \textbf{R\textbar{}P 2026}!

\begin{center}\includegraphics[width=0.9\columnwidth]{./articles/images/full_staff.jpg}\end{center}


\closearticle


\needspace{5\baselineskip}

\label{article:qiskitfallfest}

\byline{\textbf{\Large Qiskit Fall Fest}}{Sasha Levinshteyn, SIGQuantum Exec}

It’s that time of the year again: Qiskit Fall Fest is upon us! Qiskit Fall Fest is an IBM Quantum-sponsored month-long collection of quantum-related events. SIGQuantum kicked off our second ever Qiskit Fall Fest last Wednesday (October 15th) with an amazing talk on “Quantum Error Correction and Information Theory” by Sujeet Bhalerao. We had pizza, stickers, and quantum, what else could a quantum enthusiast dream to ask for? A whole \textit{month} of quantum computing talks and Qiskit workshops? Turns out the answer is a resounding yes! 

Throughout the next month, we will have a series of fascinating talks and events related to quantum computing and Qiskit (a coding language for quantum computers), culminating with an Open Source Quantum Hackathon. We invite all of you to join us to learn about quantum computing from a variety of perspectives at our upcoming events! All are welcome, regardless of background! 

The past and upcoming events are summarized in the flier on the next page. We are still trying to figure out some final details, so keep up to date by joining the SIGQuantum Discord server. We would like to highlight a few events in particular:
1. IBM Speaker on Advanced Qiskit (date TBD, likely around November 9th) - We will be hosting a spectacular speaker from \textbf{IBM} to walk us through some advanced Qiskit programming. Do you want to learn how to code a quantum computer? If so, this event is for you! (We strongly encourage all to attend the Beginner Qiskit and Intermediate Qiskit sessions before this.)
2. Undergraduate Lightning Talks (date TBD, likely around November 9th) - A number of undergraduates are going to give short talks on subjects in quantum computing they find interesting. We invite all of you to come support these undergraduates! If you’re an undergraduate interested in presenting (on literally anything related to quantum computing), please reach out to SIGQuantum exec on our Discord. 
3. Open Source Quantum Hackathon (November 15th - 17th) - Join us for the final event of Qiskit Fall Fest on November 15th! This is a chill kind of Hackathon. We’re going to split into groups and choose something open source and quantum-related, for example, the Qiskit coding language itself, to contribute to. This will be a great opportunity to learn something new and show off your skills in time for our hackathon trip next semester!

If you have any questions about these events or would like to sign up to give a lightning talk, please reach out to us on Discord (@academicweapona or on the SIGQuantum Discord server). 

As a reminder, everyone is welcome, regardless of background. Did you just hear about quantum computers for the first time and want to find out what all the buzz is about? Come to Qiskit Fall Fest! Have you been messing around in quantum computing theory or hardware and realize you need to learn some software? Come to Qiskit Fall Fest! Do you think all of this quantum computing buzz is bullshit? That’s okay, come to Qiskit Fall Fest!

We look forward to seeing all of you at our events. Here’s to an amazing second run of Qiskit Fall Fest!

\begin{center}\includegraphics[width=0.9\columnwidth]{./articles/images/QiskitFallFest.png}\end{center}


\closearticle


\needspace{5\baselineskip}

\label{article:robotics}

\byline{\textbf{\Large SIGRobotics at Cal Hacks 12.0 \& Embodied AI Hackathon}}{SIGRobotics}

Last weekend (Oct. 24-26), two SIGRobotics teams set off for San Francisco to compete in Cal Hacks 12.0 and the Embodied AI Hackathon.

\subsection*{Embodied AI Hackathon}

\begin{center}\includegraphics[width=0.9\columnwidth]{./articles/images/SIGRobotics_Seeed_Team.jpg}\end{center}
\textit{Left to right: Stephen Zhu, Hasan Al Saeedi, Keshav Badrinath, Filip Kujuwa, Himank Handa, Sanjit Kumar, Aarsh Mittal, Leo Lin}

\vspace{0.5em}

Leo Lin, Himank Handa, Keshav Badrinath, Aarsh Mittal, Filip Kujawa, Sanjit Kumar, Stephen Zhu, and Hasan Al Saeedi from SIGRobotics placed first at the Embodied AI Hackathon hosted by NVIDIA, HuggingFace, and Seeed Studio. Their winning project featured a matcha-making robot, which they named \textit{Performative}. Their project was a demonstration of embodied AI, a field that leverages novel generative AI methods to help robots learn how to perform real-world tasks.

The team engineered a setup with two SO-101 robotics arms to replicate the fine-grained motions part of the process of making matcha. They used NVIDIA’s most powerful robotics foundation model, GR00T N1.5, a VLA (vision language action model) which receives image and natural language prompts as input and outputs a corresponding action (similar to how you would chat with ChatGPT, but instead of returning a paragraph, your robot performs actions for you). The inference to run this model on the robot was executed on an NVIDIA Jetson.

Though there were many hardware challenges along the way, including one of the arms and one of the cameras completely breaking, the team adapted and overcame these, fine-tuning and running their final model for the first time 20 minutes before the demo to the judges (the second time they ran it was during the demo!). Read more about this project here: \begin{center}\begingroup\color{black}\qrcode[height=0.8in,hyperlink=false]{https://www.hackster.io/sigrobotics/embodied-ai-hackathon-submission-sigrobotics-1st-place-f0e520}\endgroup\\\vspace{0.15cm}{\small https://www.hackster.io/sigrobotics/embodied-ai-hackathon-submission-sigrobotics-1st-place-f0e520}\end{center}


\subsection*{Cal Hacks 12.0}

\begin{center}\includegraphics[width=0.9\columnwidth]{./articles/images/SIGRobotics_CalHacks_Team.jpeg}\end{center}
\textit{Left to right: Yash Yardi, William Po-Yen Chou, Tanish Mittal, Krish Konda}  

\vspace{0.5em}

Last weekend, four SIGRobotics members, Yash Yardi, William Po-Yen Chou, Tanish Mittal, and Krish Konda, took off for San Francisco to compete in the World’s Largest Collegiate Hackathon: Cal Hacks 12.0. The event brought together 2000+ competitors to the Palace of Fine Arts, with scenic views and a packed agenda for 36 hours of nonstop building, coding, and creative chaos. The team set out to build a robot that could draw what you imagined. 

They call it MARC: Marker Actuated Robotic Controller, which takes natural-language prompts and turns them into real pen-on-paper drawings. The system uses an LLM for image generation to create artwork from a user’s prompt and then converts it into precise motion commands. Then, an SO-100 robotic arm follows the commands to physically sketch the image. 

After demoing MARC at Cal Hacks, the team won second place in the robotics track! The hackathon put the team’s collaboration and design skills to the test, with Krish commenting, “This was a great experience, and developing MARC gave us inspiration to make this into something anyone can use.” See more about this project here: \begin{center}\begingroup\color{black}\qrcode[height=0.8in,hyperlink=false]{https://devpost.com/software/marc-marker-actuated-robotic-controller}\endgroup\\\vspace{0.15cm}{\small https://devpost.com/software/marc-marker-actuated-robotic-controller}\end{center}


\closearticle


\needspace{5\baselineskip}

\label{article:icpc}

\byline{\textbf{\Large UIUC ICPC Triumphs}}{Enya Chen}

This year, UIUC student team Ippatsu—\textbf{Yuuki Sawanoi, Dilhan Salgado, and Zhikun Wang}—has been crowned \textbf{ICPC North America Champions}! 

They solved 12/13 problems, placing them 1st place out of 52 of the best teams among the best schools across across North America, winning the Gold medal. Alongside this, they were the first team to solve 3 different problems, and the 1st place champions of the North America Central Division.

They proceeded onto the ICPC World finals in Baku, securing 20th out of 139 teams. They solved 8/12 problems, going against some the best teams across in the entire world, representing over 103 countries. Alongside this, they were the first team to solve problem F. 

Both coaches Professor \textbf{Mattox Beckman} and PhD student \textbf{David Zheng}, were also awarded at the ICPC World Finals, in recongition of 5+ years of ICPC coaching.

\begin{center}\includegraphics[width=0.9\columnwidth]{./articles/images/icpc.jpg}\end{center}


\closearticle


\needspace{20\baselineskip}

\label{article:sigplan}

\byline{\textbf{\Large Making Mathematics Open Source}}{Eyad Loutfi}

\begin{center}\includegraphics[width=0.9\columnwidth]{./articles/images/sigplan.png}\end{center}
    
In recent years, interactive theorem proving has revealed unexpected potential, both enhancing collaboration among professional mathematicians and making the field accessible to software engineers and those from less traditional academic backgrounds.

Many of the benefits of mathematicians adopting such technologies are obvious - digitizing a library of theorems would open it up to search and other automotive tools, which could then be used to assist in the building of more complicated modern proofs. Should we reach the point where modern research level proofs are built with or at least checked by a theorem prover, ensuring correctness would no longer be a matter of faith in the author or in the wait for peer review. 

The technology is still in its infancy in adoption by the larger mathematical community, in part since to get a proof to make sense to a computer requires consideration of many smaller details - often hand waved away or treated more informally in real life. In turn, the technology is ways off from mathematicians feeling it’s easier to work with proof assistants than without, and therefore worth the opportunity cost to learn. That being said, this comes closer to being a reality the more automation there is and the more mathematics gets digitzed. 

Great strides are already being seen - a library called mathlib for the theorem prover Lean has seen over half the standard undergrad math curriculum programmed into it the last few years (as of writing, it contains 116,770 definitions and 236,001 theorems). Several laborious proofs have also been formally verified, and last year, a notoriously difficult problem - the value of the fifth busy beaver number was proven specifically in the rocq theorem prover. Not just that, but it was done by a group that included many non-mathematicians. 

\subsection*{Background}

The busy beaver function is one which gives deep insights into computability, as the nth busy beaver is the maximum number of steps a machine with n rules can take before halting. This means if you know the nth busy beaver and your program runs for longer than that, that program is guaranteed to run forever. At first glance, this might suggest a workaround to the famously uncomputable halting problem, until one learns that this function too is uncomputable. Uncomputable here means that there cannot exist any algorithm that can take any input and spit out the corresponding output, and it shows up constantly in the theory of computation to remind us of the expressive power and limitations of what computers can and can’t do. Therefore, it is not even a given that we’ll always be able to find the next busy beaver number, and the difficulty certainly explodes to a tremendous degree with each value that has been found.


\closearticle


\needspace{5\baselineskip}

\label{article:signll}

\byline{\textbf{\Large Why NLP Matters More than Ever}}{Kush Bhardwaj}

You have probably heard of the term “natural language processing.” Maybe you think of it as a buzzword - isn’t it just ChatGPT, or a subset of artificial intelligence? However, it is a lot bigger than that. NLP is not just a subset of AI; it is the layer that makes AI usable by humans. It is the reason you are able to interact with the AI models you love and use. At its core, NLP is about teaching computers to interpret, structure, and generate language in a way that captures meaning and intention. It is not about flashy demos or the “state-of-the-art.” Rather, it is about modelling and making sense of the understandability and interactability itself. So if NLP is bigger than chatbots, what does it actually do, and why does it matter for computer scientists today?

At the practical level, NLP enables computers to extract meaning from unstructured language. A good NLP model would have to effectively manage the following from users: 
\begin{itemize}
\item The semantics of the sentence - the user intent and what the sentence actually means
\item The syntax of the sentence - the grammatical structure of a sentence to understand how words relate to each other
\item Be able to articulate and understand the output
\end{itemize}
Some examples of effective NLP models would be speech-accessibility tools, document summarization, translation models, and more. 

NLP models are also transforming many different industries, and are quickly becoming endemic to the workplace. It is used in the business industry, where it is being used to automatically process vast amounts of data, like customer review data, legal documents, and more. It is also often used in the healthcare industry - aiding in diagnostics, accelerating drug discovery by finding connections in literature, and streamlining administrative tasks like medical coding.

As these models are normalized and become endemic to the workplace, it has become more important to fundamentally understand how they work. Understanding the underlying mechanisms (e.g., transformer architectures) allows professionals to properly integrate, customize, and maintain these systems for specific company needs. Additionally, it is also crucial to realize the pitfalls as well. Models are only as good as their training data; if the data contains human biases (e.g., racial, gender), the model will perpetuate and even amplify them. This can lead to unfair or discriminatory outcomes in hiring, lending, or legal contexts. On top of that, it is not uncommon for models to sometimes lose context or “hallucinate” (confidently generating plausible-sounding but incorrect information). 

Overall, NLP is a very important layer between the human and the AI models. It is the reason AI models are able to understand human input, and vice versa. However, when it comes to AI models, the future of work means not just that the use of NLP tools will explode, but there is also a responsibility to understand their inner workings and limits.


\closearticle


\needspace{5\baselineskip}

\label{article:gamebuilders}

\byline{\textbf{\Large Gamebuilders Updates!}}{Ethan Wang}

So far, we’ve already hosted our semesterly game jam and workshops on Blender and Shaders, and are planning ahead for future workshops and our industry panel.

In terms of semester projects, our members have been hard at work, and we’re happy to say that the games are starting to take flight! We feel that our games will truly be on an island of their own in terms of quality and are hard at work on the Unity files to try to make these ideas come to life! We’re hoping to release the list of finished games ASAP for our final showcase for the semester.


\closearticle


\needspace{5\baselineskip}

\label{article:siggraph}

\byline{\textbf{\Large Update from SIGGraph!}}{Owen Siemons}

SIGGRAPH is having a great semester! We have had a handful of hands-on workshops in a wide range of topics including raytracing, Gaussian splatting, Cel shading, and Cellular Automata. Accompanying a few slides explaining the topic, we supply a Shadertoy project for every member to either mess around with or iterate on what we talked about in the workshop. At the end of each meeting, members show off what they created to the rest of the club in an interactive environment. Throughout the semester we will continue to host these workshops while also making space for project meetings.

We have started development on a spatial audio visualizer using an AR headset and a unified backend abstraction for both OpenGL and Vulkan APIs to make interfacing with them easier for any future SIGGRAPH projects. The spatial audio visualizer project presents a lot of unique challenges to it including triangulating audio to find the source, decomposing audio signals based on distances, and needing to utilize machine learning to classify noise. The backend project is unique because an abstraction from these rendering APIs is not commonly done, and when it is, it is not typically publicly accessible. This means a lot of meaningful design decisions are being made daily and every contributor is learning about many technical aspects of creating computer graphics from scratch.

We meet every Sunday at 2PM! Join our Discord (find it at https://www.acm.illinois.edu/) for more information because we would love to have you!


\closearticle


\needspace{5\baselineskip}

\label{article:sigecom}

\byline{\textbf{\Large SIGEcom Updates!}}{Bhargav Sampathkumaran}

This semester, SIGecom has been diving deep into the intersection of computer science and economics. Each week, we’ve presented on topics ranging from algorithmic trading and optional membership design to game theory in AI, dynamic pricing, and platform economics—exploring how algorithms shape the markets and digital platforms we interact with every day.

Beyond presentations, we’re shifting toward hands-on experimentation. We’ve kicked off two new projects: one focused on developing a program tailored for simulating prediction markets, and another centered around building an algorithmic trading system. These initiatives give members the chance to explore how data, modeling, and automation can come together to simulate and influence real-world markets.

A highlight this semester is Traydner, developed by Evan Doubek. Traydner is a learn-by-doing trading simulator where users can practice trading stocks, crypto, and forex using a risk-free paper wallet. It’s an exciting way for members to experiment with strategy and market behavior without financial risk.

At SIGecom, our goal is simple: to get people’s hands dirty with real projects—no experience required. Whether you’re an economist curious about coding or a CS student fascinated by markets, everyone is welcome to join and learn by building.


\closearticle

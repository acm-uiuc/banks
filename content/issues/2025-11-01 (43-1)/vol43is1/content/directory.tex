\newpage
\label{directory}
\begin{center}
\textbf{\underline{\Huge ACM @ UIUC Directory}}
\end{center}
\vspace{0.3cm}

\begin{multicols}{2}

\noindent
\begin{minipage}{\columnwidth}
\begin{center}
\includegraphics[width=0.35\columnwidth]{./logo/acm.png}
\end{center}
\vspace{0.05cm}
\subsection*{ACM}
\noindent\textbf{Chair:} Jacob Levine\\
\noindent\textbf{Vice Chair:} Sherry Long\\
\noindent\textbf{Treasurer:} Adhi Thirumala\\
\noindent\textbf{Secretary:} Krish Gangal\\
\noindent\textbf{Website:} \href{https://acm.illinois.edu}{acm.illinois.edu}\\
\noindent\textbf{Discord:} \href{https://acm.gg/discord}{acm.gg/discord}\\
\noindent\textbf{Instagram:} \href{https://instagram.com/acm.uiuc}{instagram.com/acm.uiuc}\\

{\setlength{\parindent}{1.5em}
Officers of the overarching ACM@UIUC organisation.
}
\end{minipage}

\vspace{0.3cm}

\noindent\rule{\columnwidth}{0.4pt}
\vspace{0.3cm}

\noindent
\begin{minipage}{\columnwidth}
\begin{center}
\includegraphics[width=0.35\columnwidth]{./logo/corporate_committee.png}
\end{center}
\vspace{0.05cm}
\subsection*{Corporate Committee}
\noindent\textbf{Chairs:} Adya Daruka, Akshay Vellore\\

{\setlength{\parindent}{1.5em}
The corporate team handles communication with ACM@UIUC's sponsors. Our current sponsors include PrairieLearn, Atlassian, Capital One, Citadel, RandomLabs, Visa, and KLA.
}
\end{minipage}

\vspace{0.3cm}

\noindent\rule{\columnwidth}{0.4pt}
\vspace{0.3cm}

\noindent
\begin{minipage}{\columnwidth}
\begin{center}
\includegraphics[width=0.35\columnwidth]{./logo/infrastructure_committee.png}
\end{center}
\vspace{0.05cm}
\subsection*{Infrastructure Committee}
\noindent\textbf{Chairs:} Dev Singh, Kay Rivera\\
\noindent\textbf{Website:} \href{https://infra.acm.illinois.edu}{infra.acm.illinois.edu}\\

{\setlength{\parindent}{1.5em}
The Infra Committee is responsible for managing ACM @ UIUC’s core software engineering and infrastructure efforts, both on-premises and in the cloud. Our main project is ACM core, a self-service platform for exec council members to handle event management, ticketing, access management, membership servicing, link shortening, and more.
}
\end{minipage}

\vspace{0.3cm}

\noindent\rule{\columnwidth}{0.4pt}
\vspace{0.3cm}

\noindent
\begin{minipage}{\columnwidth}
\begin{center}
\includegraphics[width=0.35\columnwidth]{./logo/academic_committee.png}
\end{center}
\vspace{0.05cm}
\subsection*{Academic Committee}
\noindent\textbf{Chairs:} Amol Shah, Yanni Zhuang\\
\noindent\textbf{Website:} \href{https://academic.acm.illinois.edu}{academic.acm.illinois.edu}\\

{\setlength{\parindent}{1.5em}
The academic committee organizes review events for CS courses and interfaces with professors to enable student success in courses.
}
\end{minipage}

\vspace{0.3cm}

\noindent\rule{\columnwidth}{0.4pt}
\vspace{0.3cm}

\noindent
\begin{minipage}{\columnwidth}
\begin{center}
\includegraphics[width=0.35\columnwidth]{./logo/social_committee.png}
\end{center}
\vspace{0.05cm}
\subsection*{Social Committee}
\noindent\textbf{Chairs:} Ashika Koripelly, Naomi Lin\\

{\setlength{\parindent}{1.5em}
The social team organizes fun events for ACM@UIUC such as picnics, activity weeks, game nights, and Happy Hour.
}
\end{minipage}

\vspace{0.3cm}

\noindent\rule{\columnwidth}{0.4pt}
\vspace{0.3cm}

\noindent
\begin{minipage}{\columnwidth}
\begin{center}
\includegraphics[width=0.35\columnwidth]{./logo/marketing_committee.png}
\end{center}
\vspace{0.05cm}
\subsection*{Marketing Committee}
\noindent\textbf{Chairs:} Jasmine Liu, Veda Fernandes\\

{\setlength{\parindent}{1.5em}
The marketing team organizes social media and branding for ACM@UIUC.
}
\end{minipage}

\vspace{0.3cm}

\noindent\rule{\columnwidth}{0.4pt}
\vspace{0.3cm}

\noindent
\begin{minipage}{\columnwidth}
\begin{center}
\includegraphics[width=0.35\columnwidth]{./logo/mentorship_committee.png}
\end{center}
\vspace{0.05cm}
\subsection*{Mentorship Committee}
\noindent\textbf{Chairs:} Alice Fan, Mitali Ahlawat\\

{\setlength{\parindent}{1.5em}
The mentorship team organizes mentorship programs for ACM@UIUC, welcoming everyone to CS @ Illinois.
}
\end{minipage}

\vspace{0.3cm}

\noindent\rule{\columnwidth}{0.4pt}
\vspace{0.3cm}

\noindent
\begin{minipage}{\columnwidth}
\begin{center}
\includegraphics[width=0.35\columnwidth]{./logo/reflections_projections.jpg}
\end{center}
\vspace{0.05cm}
\subsection*{Reflections \textbar{} Projections}
\noindent\textbf{Chairs:} Cole Jordan, Shreenija Daggavolu\\
\noindent\textbf{Website:} \href{https://reflectionsprojections.org/}{reflectionsprojections.org/}\\

{\setlength{\parindent}{1.5em}
We provide a forum to share and learn about progress in computer science, with industry and academia tech talks, workshops and events for attendees, Mechmania, and Diversity × Tech.
}
\end{minipage}

\vspace{0.3cm}

\noindent\rule{\columnwidth}{0.4pt}
\vspace{0.3cm}

\noindent
\begin{minipage}{\columnwidth}
\begin{center}
\includegraphics[width=0.35\columnwidth]{./logo/hackillinois.png}
\end{center}
\vspace{0.05cm}
\subsection*{HackIllinois}
\noindent\textbf{Chairs:} Lucy Wu, Sada Challa\\
\noindent\textbf{Website:} \href{https://hackillinois.org}{hackillinois.org}\\

{\setlength{\parindent}{1.5em}
HackIllinois is UIUC's premier collegiate hackathon. With over 1000 attendees and 50 mentors in 2019, the hackathon has become one of the largest and most well-regarded in the nation.
}
\end{minipage}

\vspace{0.3cm}

\noindent\rule{\columnwidth}{0.4pt}
\vspace{0.3cm}

\noindent
\begin{minipage}{\columnwidth}
\begin{center}
\includegraphics[width=0.35\columnwidth]{./logo/sigpwny.png}
\end{center}
\vspace{0.05cm}
\subsection*{SIGPwny}
\noindent\textbf{Chairs:} Cameron Asher, Emma Hartman, Minh Duong, Michael Khalaf, Nikhil Date, Sam Ruggerio\\
\noindent\textbf{Meetings:} Sundays, 5:00 PM--6:00 PM, SC 1404\\
\noindent\phantom{\textbf{Meetings:} }Thursdays, 6:00 PM--7:00 PM, DCL 1310\\
\noindent\textbf{Website:} \href{https://sigpwny.com}{sigpwny.com}\\
\noindent\textbf{Discord:} \href{https://discord.gg/cWcZ6a9}{discord.gg/cWcZ6a9}\\

{\setlength{\parindent}{1.5em}
hallo minh please do the thing here thanks :)
}
\end{minipage}

\vspace{0.3cm}

\noindent\rule{\columnwidth}{0.4pt}
\vspace{0.3cm}

\noindent
\begin{minipage}{\columnwidth}
\begin{center}
\includegraphics[width=0.35\columnwidth]{./logo/sigchi.png}
\end{center}
\vspace{0.05cm}
\subsection*{SIGCHI}
\noindent\textbf{Chairs:} Aditi Shrivastava, Ethan Nguyen, Josephine Lee, Maya Zubak\\
\noindent\textbf{Meetings:} Mondays, 5:00 PM--4:20 PM, SC 2405\\
\noindent\textbf{Website:} \href{https://sigchi.acm.illinois.edu/}{sigchi.acm.illinois.edu/}\\
\noindent\textbf{Discord:} \href{https://discord.gg/XRShsPCAQ3}{discord.gg/XRShsPCAQ3}\\

{\setlength{\parindent}{1.5em}
SIGCHI is the special interest group for Human-Computer Interaction. We explore ways to design and make technology that solves problems or enhances the human experience in small ways. Every year, we focus on designing and building a project that incorporates principles of human-centered design and present it at Engineering Open House!
}
\end{minipage}

\vspace{0.3cm}

\noindent\rule{\columnwidth}{0.4pt}
\vspace{0.3cm}

\noindent
\begin{minipage}{\columnwidth}
\begin{center}
\includegraphics[width=0.35\columnwidth]{./logo/gamebuilders.png}
\end{center}
\vspace{0.05cm}
\subsection*{GameBuilders}
\noindent\textbf{Chairs:} Ethan Wang\\
\noindent\textbf{Meetings:} Wednesdays, 7:00 PM--9:00 PM, SC 1302\\
\noindent\textbf{Website:} \href{https://gamebuilders.acm.illinois.edu/}{gamebuilders.acm.illinois.edu/}\\
\noindent\textbf{Discord:} \href{https://discord.com/invite/2rND6FT}{discord.com/invite/2rND6FT}\\

{\setlength{\parindent}{1.5em}
Gamebuilders is a SIG focused on game development. We host game jams, game dev workshops and sometimes industry panels, but the main focus of the club is our semester projects, which are proposed and worked on by our members. Project leads will then present their project at our MVP and Final showcase to the rest of the club and any attendees! Gamebuilders works on all aspects of game development, including Art, Music, and Game Design, and is open to all skill levels, so don't be afraid to join even with no experience!
}
\end{minipage}

\vspace{0.3cm}

\noindent\rule{\columnwidth}{0.4pt}
\vspace{0.3cm}

\noindent
\begin{minipage}{\columnwidth}
\begin{center}
\includegraphics[width=0.35\columnwidth]{./logo/sigaida.png}
\end{center}
\vspace{0.05cm}
\subsection*{SIGAIDA}
\noindent\textbf{Chairs:} Kaavya Mahajan, Rishi Mulchandani\\
\noindent\textbf{Meetings:} Mondays, 6:00 PM--7:30 PM, Everitt 1306\\
\noindent\textbf{Website:} \href{https://aida.acm.illinois.edu}{aida.acm.illinois.edu}\\
\noindent\textbf{Discord:} \href{https://acm.gg/aida_discord}{acm.gg/aida\_discord}\\

{\setlength{\parindent}{1.5em}
Ever wondered how AI actually works - or better yet, how to make it work? That’s what we do at AIDA. We explore the world of artificial intelligence and data analytics through hands-on projects, workshops, and demos that turn abstract concepts into real results.

Our beginner track, MLScope, walks you through the fundamentals of AI and machine learning with guided sessions and practical mini-projects - from predicting housing prices to analyzing massive textual datasets with large language models. For those ready to level up, our project track dives into real-world AI applications. This semester’s lineup includes six active projects, covering everything from sustainable recycling assistants to financial reasoning and sports analytics.

Whether you’re brand new to AI or already training deep learning models in your sleep, AIDA’s got something for you. Come learn, build, and collaborate with a group of curious minds who think data is cool (because it is).

Mondays at 6. Be there :D
}
\end{minipage}

\vspace{0.3cm}

\noindent\rule{\columnwidth}{0.4pt}
\vspace{0.3cm}

\noindent
\begin{minipage}{\columnwidth}
\begin{center}
\includegraphics[width=0.35\columnwidth]{./logo/siggraph.png}
\end{center}
\vspace{0.05cm}
\subsection*{SIGGRAPH}
\noindent\textbf{Chairs:} Eero Dunham, John Barry, Owen Siemons\\
\noindent\textbf{Meetings:} Sundays, 2:00 PM--3:00 PM, SC 1302\\
\noindent\textbf{Website:} \href{https://siggraph.acm.illinois.edu/}{siggraph.acm.illinois.edu/}\\
\noindent\textbf{Discord:} \href{https://discord.com/invite/a5U333fNMX}{discord.com/invite/a5U333fNMX}\\

{\setlength{\parindent}{1.5em}
SIGGRAPH is UIUC’s main RSO about all things computer graphics. We host breadth talks, demos, project development, alumni talks, research paper deep-dives, and social events. Whether you are a beginner or experienced beyond your years, there is a place for you in SIGGRAPH to express your interests and meet like-minded people.
}
\end{minipage}

\vspace{0.3cm}

\noindent\rule{\columnwidth}{0.4pt}
\vspace{0.3cm}

\noindent
\begin{minipage}{\columnwidth}
\begin{center}
\includegraphics[width=0.35\columnwidth]{./logo/icpc.png}
\end{center}
\vspace{0.05cm}
\subsection*{ICPC}
\noindent\textbf{Chair:} Canchen Li, Daniel Zhang, Enya Chen, Johnathan Tong, Ryan To\\
\noindent\textbf{Meetings:} Thursdays, 5:00 PM--7:00 PM, SC 1302\\
\noindent\textbf{Website:} \href{http://icpc.cs.illinois.edu/}{icpc.cs.illinois.edu/}\\
\noindent\textbf{Discord:} \href{http://acm.gg/icpc_discord}{acm.gg/icpc\_discord}\\

{\setlength{\parindent}{1.5em}
SIG-ICPC is a special interest group for the International Collegiate Programming Contest (ICPC) that runs under the ACM chapter at UIUC. The ICPC is an algorithmic programming contest for college students. It is the oldest, largest, and most prestigious programming contest in the world. More than 50,000 students participate in three-person teams representing more than 3,000 universities.

Here we foster a friendly environment for students to practice and train for not only ICPC, but also to improve their programming and problem solving skills. These skills prove to be beneficial in technical courses, programming interviews, and real life applications.

In the past 17 years, our UIUC has advanced to World Finals 16 times. Let's continue this legacy!

For more information, visit our website at \href{https://icpc.cs.illinois.edu}{icpc.cs.illinois.edu} or join our Discord server at \href{https://discord.gg/icpc-uiuc}{discord.gg/icpc-uiuc}.
}
\end{minipage}

\vspace{0.3cm}

\noindent\rule{\columnwidth}{0.4pt}
\vspace{0.3cm}

\noindent
\begin{minipage}{\columnwidth}
\begin{center}
\includegraphics[width=0.35\columnwidth]{./logo/sigmusic.png}
\end{center}
\vspace{0.05cm}
\subsection*{SIGMusic}
\noindent\textbf{Chairs:} Aslan Wang, Emmett Quan, William Lei\\
\noindent\textbf{Meetings:} Tuesdays, 6:30 PM--8:00 PM, CS 1304\\
\noindent\textbf{Website:} \href{http://sigmusic.acm.illinois.edu/}{sigmusic.acm.illinois.edu/}\\
\noindent\textbf{Discord:} \href{http://acm.gg/sigmusic_discord}{acm.gg/sigmusic\_discord}\\

{\setlength{\parindent}{1.5em}
SIGMusic focuses on discussions and projects surrounding digital audio processing, audio hardware, and electroacoustic techniques for the creation of music. Under the leadership of Emmett Quan (CS + Music, '28) and William Lei (CS + Music, '28), the organization is currently creating digital guitar pedals using the Daisy framework. All are welcome to come to weekly Tuesday evening meetings, (6:30-8:00pm in Siebel CS 1304), regardless of experience with computers or audio! Join the Discord server to stay in the loop.
}
\end{minipage}

\vspace{0.3cm}

\noindent\rule{\columnwidth}{0.4pt}
\vspace{0.3cm}

\noindent
\begin{minipage}{\columnwidth}
\begin{center}
\includegraphics[width=0.35\columnwidth]{./logo/glug.png}
\end{center}
\vspace{0.05cm}
\subsection*{GLUG}
\noindent\textbf{Chairs:} Jenna Fligor, Krishnan Shankar\\
\noindent\textbf{Meetings:} Wednesdays, 6:00 PM--7:00 PM, SC 1302\\
\noindent\textbf{Website:} \href{https://lug.acm.illinois.edu/}{lug.acm.illinois.edu/}\\
\noindent\textbf{Discord:} \href{https://discord.gg/sWD3zxPyc2}{discord.gg/sWD3zxPyc2}\\
\noindent\textbf{Matrix:} \href{https://matrix.to/#/#gnulug:matrix.org}{matrix.to/\#/\#gnulug:matrix.org}\\

{\setlength{\parindent}{1.5em}
GLUG is dedicated to learning about the fundamentals of operating systems, with an emphasis on Linux and UNIX-based operating systems. We talk about a wide variety of concepts - some examples include self-hosting, service managers, Linux distros, networking, privacy/cryptography, and general news in the free and open source software (FOSS) community. If any of these topics sound interesting to you, join us on Wednesdays from 6-7pm at Siebel CS 1302! Everyone is welcome, all talks are beginner-friendly, and you definitely don't have to use Linux to show up. Our meeting schedule and Discord/Matrix servers are linked on our website, at \href{https://lug.acm.illinois.edu}{https://lug.acm.illinois.edu}.
}
\end{minipage}

\vspace{0.3cm}

\noindent\rule{\columnwidth}{0.4pt}
\vspace{0.3cm}

\noindent
\begin{minipage}{\columnwidth}
\begin{center}
\includegraphics[width=0.35\columnwidth]{./logo/signll.png}
\end{center}
\vspace{0.05cm}
\subsection*{SIGNLL}
\noindent\textbf{Chairs:} Kush Bhardwaj, Ryan Varghese, Saad Ahmad, Vinay Rajagopalan\\
\noindent\textbf{Discord:} \href{https://discord.gg/wwYeewYkCG}{discord.gg/wwYeewYkCG}\\

{\setlength{\parindent}{1.5em}
Hello, we’re SIGNLL, the SIG dedicated to NLP! Right now, our officers are Kush Bhardwaj, Saad Ahmad, Ryan Varghese, and Vinay Rajagopalan. Each semester, we give our members an introduction into the world of NLP by helping them create the NLP-related projects that they've always wanted to make! And at the time of writing, we’ve just started working on those projects! Some of the problems our members are working on include making small language models for specific tasks, a sports betting assistant, a resume parser, and more! If you’re worried about missing out, boy, do I have great news for you! Even though we’ve already started working on projects, you can still join! Just show up to our meetings (6-7 pm every Monday, in the Siebel Center for CS room 1304!) If you want, you can join one of the existing projects, or, if you have your own idea, feel free to form your own group and work on your own NLP-related project!
}
\end{minipage}

\vspace{0.3cm}

\noindent\rule{\columnwidth}{0.4pt}
\vspace{0.3cm}

\noindent
\begin{minipage}{\columnwidth}
\begin{center}
\includegraphics[width=0.35\columnwidth]{./logo/sigma.png}
\end{center}
\vspace{0.05cm}
\subsection*{SIGma}
\noindent\textbf{Chairs:} Alex Broihier, Sasha Levinshteyn, Franklin Zhang, Ian Chen, Porter Shawver, Sam Ruggerio\\
\noindent\textbf{Meetings:} Mondays, 6:00 PM--7:00 PM, SC 1302\\
\noindent\textbf{Website:} \href{https://cstheory.org/}{cstheory.org/}\\
\noindent\textbf{Discord:} \href{https://discord.gg/Sxf3h3pBbv}{discord.gg/Sxf3h3pBbv}\\

{\setlength{\parindent}{1.5em}
SIGma is the Special Interest Group for Math and Algorithms here at UIUC. We cover everything and anything in the CS theory field. Most of our regular meetings consist of members presenting or discussing topics they’re interested in. All backgrounds welcome!
}
\end{minipage}

\vspace{0.3cm}

\noindent\rule{\columnwidth}{0.4pt}
\vspace{0.3cm}

\noindent
\begin{minipage}{\columnwidth}
\begin{center}
\includegraphics[width=0.35\columnwidth]{./logo/sigquantum.png}
\end{center}
\vspace{0.05cm}
\subsection*{SIGQuantum}
\noindent\textbf{Chairs:} Sasha Levinshteyn, George Huebner, River Chen, Shreyes Bharat\\
\noindent\textbf{Meetings:} Wednesdays, 6:00 PM--7:00 PM, SC 1304\\
\noindent\textbf{Website:} \href{https://sigquantum.com}{sigquantum.com}\\
\noindent\textbf{Discord:} \href{https://discord.gg/PmaXeHPaFs}{discord.gg/PmaXeHPaFs}\\

{\setlength{\parindent}{1.5em}
SIGQuantum is the Special Interest Group for Quantum Computing and Information here at UIUC. We cover everything and anything in the quantum computing field from quantum hardware to quantum software to quantum theory. Most of our regular meetings consist of members presenting or discussing topics they’re interested in. All backgrounds welcome!
}
\end{minipage}

\vspace{0.3cm}

\noindent\rule{\columnwidth}{0.4pt}
\vspace{0.3cm}

\noindent
\begin{minipage}{\columnwidth}
\begin{center}
\includegraphics[width=0.35\columnwidth]{./logo/sigecom.png}
\end{center}
\vspace{0.05cm}
\subsection*{SIGecom}
\noindent\textbf{Chairs:} Bhargav Sampathkumaran\\
\noindent\textbf{Meetings:} Mondays, 6:00 PM--7:20 PM, SC 1104\\
\noindent\textbf{Discord:} \href{https://acm.gg/ecom_discord}{acm.gg/ecom\_discord}\\

{\setlength{\parindent}{1.5em}
SIGecom is ACM’s Special Interest Group in Economics and Computation. At SIGecom, our goal is simple: to get people’s hands dirty with real projects—no experience required. Whether you’re an economist curious about coding or a CS student fascinated by markets, everyone is welcome to join and learn by building.
}
\end{minipage}

\vspace{0.3cm}

\noindent\rule{\columnwidth}{0.4pt}
\vspace{0.3cm}

\noindent
\begin{minipage}{\columnwidth}
\begin{center}
\includegraphics[width=0.35\columnwidth]{./logo/sigplan.png}
\end{center}
\vspace{0.05cm}
\subsection*{SIGPLAN}
\noindent\textbf{Chairs:} Eyad Loutfi, Ethan Zhang\\
\noindent\textbf{Meetings:} Wednesdays, 5:00 PM--6:00 PM, Siebel 1302\\
\noindent\textbf{Discord:} \href{https://acm.gg/sigplan_discord}{acm.gg/sigplan\_discord}\\

{\setlength{\parindent}{1.5em}
Interested in having a programming language debate? Want to discuss your favorite formal verifier? Maybe even learn some category theory while you’re at it?

Sigplan is the place to casually meet other fellow PL enthusiasts (often to some pizza : D) and discuss related topics, be it in compilers, functional programming, formal verification, or mathematical logic and type theory (with its endless applications to both programming and math, or even the tangentially related but deep discussion of what mathematics even is). If these topics seem daunting, don’t worry since all are welcome to join with or without background, and we encourage anyone to come learn!

Besides the opportunity for socials and presentations on intriguing and requested topics, anyone is further welcome to give their own presentations. Whether to share a related topic they’re passionate about, or as motivation to encourage self learning and practice communication of said topics.

Sigplan meets biweekly, with the next meeting on November 12th (will be on the history of programming languages!). We encourage joining our discord to stay up to date with announcements, (lecture and guest speaker days will also be recorded and posted there). If there are any questions, feel free to reach out to one of our two chairs either by email or discord.
}
\end{minipage}

\vspace{0.3cm}

\noindent\rule{\columnwidth}{0.4pt}
\vspace{0.3cm}

\noindent
\begin{minipage}{\columnwidth}
\begin{center}
\includegraphics[width=0.35\columnwidth]{./logo/sigpolicy.png}
\end{center}
\vspace{0.05cm}
\subsection*{SIGPolicy}
\noindent\textbf{Chairs:} Daniel Elliott, Pranav Rajpal, Camille Wu\\
\noindent\textbf{Meetings:} Tuesdays, 7:00 PM--8:00 PM, Everitt 3117\\
\noindent\textbf{Website:} \href{https://sigpolicy.acm.illinois.edu/}{sigpolicy.acm.illinois.edu/}\\
\noindent\textbf{Discord:} \href{https://discord.gg/gKjMH54YBF}{discord.gg/gKjMH54YBF}\\

{\setlength{\parindent}{1.5em}
At SigPolicy, we aim to bring the tech events, news, and debates of the world to everyone at Illinois — that means you! Our meetings cover critical topics in tech, from chip control to social media funneling to online gambling. We do the research, so all you have to do is show up, listen to us for a few minutes, and dive into the discussion — no previous knowledge or experience with the topic needed! Coming up, we are having a social on the 4th, with Manolo’s and Mafia, to which everyone is welcome, and then diving into malicious AI — where we see it, what it means, and what we can do about it — on the 18th!
}
\end{minipage}

\vspace{0.3cm}

\noindent\rule{\columnwidth}{0.4pt}
\vspace{0.3cm}

\noindent
\begin{minipage}{\columnwidth}
\begin{center}
\includegraphics[width=0.35\columnwidth]{./logo/sigarch.png}
\end{center}
\vspace{0.05cm}
\subsection*{SIGARCH}
\noindent\textbf{Chairs:} Alex Gallagher, Pradyun Narkadamilli, Pratyay Gopal Reddy Rudravaram\\
\noindent\textbf{Meetings:} Wednesdays, 5:00 PM--6:00 PM, ECEB 3015\\
\noindent\textbf{Website:} \href{https://sigarch.net/}{sigarch.net/}\\
\noindent\textbf{Discord:} \href{https://discord.gg/Mx8R389hWz}{discord.gg/Mx8R389hWz}\\

{\setlength{\parindent}{1.5em}
SIGARCH at the University of Illinois Urbana-Champaign is a student-run group dedicated to exploring all things computer architecture — from processors and memory hierarchies to emerging trends in hardware design. We bring together students passionate about understanding and building computing systems through weekly discussions, paper reviews, and hands-on workshops. Whether you're diving into your first architecture project or already designing custom chips, SIGARCH provides a collaborative space to learn, share ideas, and grow alongside others who love hardware.
}
\end{minipage}

\vspace{0.3cm}

\noindent\rule{\columnwidth}{0.4pt}
\vspace{0.3cm}

\noindent
\begin{minipage}{\columnwidth}
\begin{center}
\includegraphics[width=0.35\columnwidth]{./logo/sigrobotics.png}
\end{center}
\vspace{0.05cm}
\subsection*{SIGRobotics}
\noindent\textbf{Chair:} Leo Lin, Manav Chandaka, Reid Faistl, Saketh Kantipudi\\
\noindent\textbf{Helper:} Advait Patel, Gloria Wang\\
\noindent\textbf{Meetings:} Saturdays, 1:00 PM--3:00 PM, SC 2405 and 1131\\
\noindent\textbf{Website:} \href{https://sigrobotics.acm.illinois.edu/}{sigrobotics.acm.illinois.edu/}\\
\noindent\textbf{Discord:} \href{https://discord.gg/Rj75e5qGT3}{discord.gg/Rj75e5qGT3}\\

{\setlength{\parindent}{1.5em}
SIGRobotics is like an open robotics lab for undergraduates! We have 40+ members across 9+ projects that include developing our own mobile robots, training robot arms with machine learning, exploring brain-computer interface applications, as well as projects tied to research-based competitions such as F1Tenth and RoboCup. We also attend a variety of robotics hackathons, and host recruiting events with robotics or robotics-adjacent companies including Neuralink this fall and General Biological last spring. 

Many of our projects do not require any prior experience, and we’re open to all majors, so feel free to come and learn about robotics! We also recently moved in to our new lab space in room 1131 in Siebel CS, and frequently host open-hours where members can come in and work on their projects. Join our discord to learn more!
}
\end{minipage}

\vspace{0.3cm}

\noindent\rule{\columnwidth}{0.4pt}
\vspace{0.3cm}

\noindent
\begin{minipage}{\columnwidth}
\begin{center}
\includegraphics[width=0.35\columnwidth]{./logo/sigtricity.png}
\end{center}
\vspace{0.05cm}
\subsection*{SIGtricity}
\noindent\textbf{Chairs:} Ansh Verma, Noah Rockoff\\
\noindent\textbf{Meetings:} Wednesdays, 6:00 PM--7:00 PM, ECEB 4070\\
\noindent\textbf{Website:} \href{https://sigtricity.acm.illinois.edu/}{sigtricity.acm.illinois.edu/}\\
\noindent\textbf{Discord:} \href{https://discord.gg/4yRWQuzhEV}{discord.gg/4yRWQuzhEV}\\

{\setlength{\parindent}{1.5em}
Sigtricity is the first SIG specializing in hardware and electrical engineering, while also incorporating CS and programming into hardware. We aim to bridge the gap between students who want to take on projects to build their resume and skill set, but do not know where to start. We provide explanations and directions that can be understood by any major or level of experience and provide all the parts. All you need to do is show up and be willing to learn.
}
\end{minipage}

\vspace{0.3cm}

\end{multicols}